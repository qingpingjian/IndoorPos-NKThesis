% -*- coding: utf-8 -*-

%\makeschapterhead{致谢}
\chapter*{致谢}

{\fangsong 依稀记得2014年初秋博士入学时自己难以掩饰的喜悦之情以及对未来的企盼,新开湖畔的背影脚步匆匆,马蹄湖上的荷花谢了又开,转眼已是将近五年,这期间,有过明日复明日的拖沓与懒惰,亦有过只争朝夕的坚持与奋发。值此博士论文完稿之际,谨对给予我帮助和支持的老师、同学和家人致以衷心的感谢。}

{\fangsong 首先,感谢我的导师张建忠教授,本文的研究工作是在张老师的悉心指导下完成的,从论文选题、资料查找、开题报告、研究思路和成果总结等各个方面,张老师都给予了我许多宝贵的意见。张老师,谆谆教诲,亦师亦长,在攻读博士学位期间,无论在治学还是在为人上都让我受益良多。}

{\fangsong 其次,感谢徐敬东教授,徐老师精深的专业知识和认真的科研作风一直影响着实验室的每一位学生。感谢吴英副教授,吴老师工作上一丝不苟,生活中平易随和,是一位值得信赖的老师。感谢张玉副教授,张老师旺盛的科研热情让人敬佩,在张老师的带领下,得以参与许多国家重大科研项目的申请和研究,极大地拓宽了我的学术视野。}

{\fangsong 特别感谢许昱玮副教授和蒲凌君老师,两位年轻老师都面临着极大的学术压力,但仍不厌其烦,或帮我思考科研规划,或帮我理清研究思路,或帮我修改学术论文,在生活和科研上为我提供了许多无私的帮助。}

{\fangsong 感谢林进挚、林安华、王昌海、于博文师兄,感谢实验室已经毕业和尚未毕业的师弟师妹们,独学而无友,则孤陋而寡闻,在求学的路上同你们一起前行,幸甚至哉。}

{\fangsong 最后,必须要感谢的是我的家人,是你们包容、鼓励和支持,才让我坚持至今。感谢我的父亲和母亲,二老都已过耳顺之年,白发渐多,皱纹日深,作为儿子不能尽孝于膝前,心中多有愧疚,谨以此文献给你们,祝愿二老身体健康。感谢我的姐姐和姐夫,从我博士入学的那日起,你们不仅担负了照顾父母的重任,还时时敦促我的学业,祝你们家庭和美,事事顺心。感谢我的女朋友,平平淡淡中已经陪我走过多年,愿与你一直走到天涯海角。}
