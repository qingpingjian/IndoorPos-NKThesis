% -*- coding: utf-8 -*-


\chapter*{个人简历、在学期间发表的学术论文与研究成果}
\section*{个人简历}
于文平,男,1986年02月01日出生。2004年9月至2008年6月就读于南开大学信息技术与科学学院计算机科学与技术专业,获工学学士学位;2008年9月至2011年6月就读于南开大学信息技术与科学学院学院计算机科学与技术专业,获工学硕士学位;2014年9月至2019年6月就读于南开大学计算机与控制工程学院(现改为网络空间安全学院)计算机科学与技术专业,攻读博士学位,研究方向为移动计算。
\section*{发表论文}
\begin{enumerate}
\renewcommand{\labelenumi}{[\theenumi]}
\item Yu Wenping, Zhang Jianzhong, Xu Jingdong and Xu Yuwei. Motion Trajectory Sequence-Based Map Matching Assisted Indoor Autonomous Mobile Robot Positioning[C]. ICA3PP. Springer, 2018. (CCF C类,已录用未发表)
\item Yu Wenping, Zhang Jianzhong, Xu Jingdong and Xu Yuwei. An Accurate Indoor Map Matching Algorithm Based on Activity Detection and Crowdsourced Wi-Fi[J]. Sci China Tech Sci. 2018. (SCI 三区,已录用未发表)
\item Yu Wenping, Xu Yuwei, Zhang Jianzhong, et al. A Smartphone-Based Online Pedestrian Positioning Approach for Both Structured and Open Indoor Spaces[C]. ISPA. IEEE, 2017. (CCF C类)
\item Yu Wenping, Zhang Jianzhong and Wang Changhai. An on-demand approach for indoor localization based on crowdsourced Wi-Fi fingerprints[C]. ICCSNT. IEEE, 2015. (EI)
\item Fu Ningjia, Zhang Jianzhong, Yu Wenping and Wang Changhai. Crowdsourcing-based wifi fingerprint update for indoor localization[C]. Proceedings of the ACM Turing 50th Celebration Conference-China. ACM, 2017. (EI)
\end{enumerate}
\section*{参与项目}
\begin{enumerate}
\renewcommand{\labelenumi}{[\theenumi]}
\item 2017年6月 - 至今,基于脑电波的射击校正实验平台
\item 2016年11月 - 2018年5月,非标准和加密VOIP协议的流量识别
\item 2009年9月 - 2011年7月,战地伤员救治卫勤训练系统
\item 2008年11月 - 2009年6月,开放式调度管理系统
\item 2008年7月 - 2008年12月,加密算法生成和校验系统
\end{enumerate}
%\section*{获得奖励}
%2020年xx月xxxx奖学金